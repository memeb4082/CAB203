\documentclass{book/custombook}
\unitcode{CA203}
\unitname{Discrete Structures}
\unitcoordinator{Matthew McKague}
\author{Dinal Atapattu}
\begin{document}
    \maketitle
    \chapter{Sets}
        \section{Set Theory}
            \begin{itemize}
                \item A set S is a collection of items (elements)
                \item Unordered and unique
                \item One basic property, \textit{membership}:
                    \begin{itemize}
                        \item $x \in S$ x is in S
                        \item $x \notin S$ x is not in S
                    \end{itemize}
                \item For $x \in S$ we also say x is an element/member of S
            \end{itemize}
            \subsection{Defining Sets}
                Sets can be defined by the following methods
                \begin{itemize}
                    \item Listing elements
                        \begin{align*}
                            \text{SMALLPRIMES} = {1,3,5,7}
                        \end{align*}
                    \item Setbuilder Notation
                        \begin{align*}
                            \text{SQUARES} = \left\{ x \in \mathrm{Z} : x = y^2 : \text{ for some } y \in \mathrm{Z} \right\}
                        \end{align*}
                    \item Implied patterns
                        \begin{align*}
                            \text{EVENS} = {2,4,6,8}
                        \end{align*}
                        \textcolor{red}{Implied conditions are bad as they are ambigious ($6 \in {2,4,...}$ or 
                        $6 \notin {2,4,...}$ depending if condition is even numbers or powers of 2)}
                \end{itemize}
            \subsection{Membership}
               $x \in S$ if 
               \begin{itemize}
                   \item x is in the list if S is given explicitly
                       \subitem $1 \in {1,2,3}$ because it exists in the list
                   \item x satisfies the conditions for S if given in setbuilder notation
                        \subitem $12 \in \left\{ x \in \mathrm{Z} : x|60\right\}$ because 12 is an intger that divides
                        60
                   \item x satisfies the implied condition for the pattern
                       \subitem $15 \in \left\{1,3,5,...\right\}$ because it follows the implied condition (odd numbers)
               \end{itemize}
            \subsection{Equality of Sets}
                Two sets are equal if they contain the same elements
                    \begin{align*}
                        S = T \Rightarrow x \in S \cap T
                    \end{align*}
            \subsection{Subsets}
                \begin{figure}[H]
                    \centering
                    \begin{flalign*}
                        \intertext{A is a subset of B if and only if every x in A exists in B (all the elements in A
                        are in B)}
                        A \subseteq B \Leftrightarrow {\forall x : x \in A \Rightarrow x \in B}
                        \intertext{A proper subset of B is a subset of B that is not equal to B}
                        A \subset B \equiv A \subseteq B \wedge A \neq B
                    \end{flalign*}
                    \caption{Improper and Proper Subset definition}
                \end{figure}
                \begin{figure}[H]
                    \centering
                    \begin{flalign*}
                        \intertext{The collection of all subsets of set A is defined as the power set of set A}
                        2^{A} = \mathcal{P}\left(A\right) = \left\{ x | x \subseteq A \right\}
                        \intertext{For example:}
                        \mathcal{P}\left(\varnothing\right\) = \left\{ \varnothing \right\}\\
                        \mathcal{P}\left(\left\{ a \right\}\right) = \left\{ \varnothing, \left\{ a \right\} \right\}\right)
                    \end{flalign*}
                    \caption{Power Set Definition}
                \end{figure}
\end{document}
