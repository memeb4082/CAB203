\documentclass{book/custombook}
\unitcode{CA203}
\unitname{Discrete Structures}
\unitcoordinator{Matthew McKague}
\author{Dinal Atapattu}
\begin{document}
    \maketitle
    \chapter{Sets}
        \section{Set Theory}
            \begin{itemize}
                \item A set S is a collection of items (elements)
                \item Unordered and unique
                \item One basic property, \textit{membership}:
                    \begin{itemize}
                        \item $x \in S$ x is in S
                        \item $x \notin S$ x is not in S
                    \end{itemize}
                \item For $x \in S$ we also say x is an element/member of S
            \end{itemize}
            \subsection{Defining Sets}
                Sets can be defined by the following methods
                \begin{itemize}
                    \item Listing elements
                        \begin{align*}
                            \text{SMALLPRIMES} = {1,3,5,7}
                        \end{align*}
                    \item Setbuilder Notation
                        \begin{align*}
                            \text{SQUARES} = \left{ x \in \mathrm{Z} : x = y^2 : \text{for some} y \in \mathrm{Z} \right}
                        \end{align*}
                    \item Implied patterns
                        \begin{align*}
                            \text{EVENS} = {2,4,6,8}
                        \end{align*}
                \end{itemize}
            \subsection{Membership}
                
\end{document}
